\documentclass[a4paper,12pt]{report}
\usepackage[top=1cm, left=1cm, right=1cm, bottom=1.5cm]{geometry}
\usepackage[T1]{fontenc}
\usepackage{hyperref,graphicx,amsmath,pdfpages,float,multirow}
\begin{document}
\begin{center}
    \textbf{CHE313A: Mass Transfer \& Its Applications\\Assignment}
\end{center}
Date:\\
Due:
\begin{enumerate}
    \item (a) In a binary mixture containing components A and B, the relative volatility of A with respect to B is 2.5 when mole fractions are used. The molecular weights of A and B are 78 and 92 respectively. If the compositions are however expressed in mass fractions, what will be the value of relative volatility?\\[3 mm]
    (b) The vapor pressure data for n-Hexane (A) and n-Octane (B) system is given below. Compute the equilibrium data and relative volatility for the system at a total pressure of 101.32 kPa.
    \begin{table}[H]
    \centering
    \begin{tabular}{|c|c|c|}
    \hline
    \multirow{2}{*}{\textbf{\begin{tabular}[c]{@{}c@{}}Temperature\\ T ($^{\circ}$C)\end{tabular}}} & \multicolumn{2}{c|}{\textbf{Vapour pressure (kPa)}} \\ \cline{2-3} 
      & \textbf{n-Hexane, \textit{P}$_\text{A}^\text{s}$} & \textbf{n-Octane, \textit{P}$_\text{B}^\text{s}$} \\ \hline
    68.7  & 101.32                & 16.1                  \\ \hline
    79.4  & 136.7                 & 23.1                  \\ \hline
    93.3  & 197.3                 & 37.1                  \\ \hline
    107.2 & 284.0                 & 57.9                  \\ \hline
    125.7 & 456.0                 & 101.32                \\ \hline
    \end{tabular}%
    \end{table}
    
    \item The vapour pressures of acetone (A) and acetonitrile (B) can be evaluated by the Antoine equations
    $$ ln \  \text{\textit{P}}_\text{A}^S = 14.5463 - \frac{2940.46}{T-35.93}$$
    $$ln \  \text{\textit{P}}_\text{B}^S = 14.2724 - \frac{2945.47}{T-49.15}$$
    \\where T is in K and \textit{P}$^\text{s}$ is in kPa. Assuming that the solutions formed by these are ideal, calculate\\
    (a) x$_\text{A}$ and y$_\text{A}$ at 327 K and 65 kPa\\
    (b) T and y$_\text{A}$ at 65 kPa and x$_\text{A}$ = 0.4\\
    (c) P and y$_\text{A}$ at 327 K and x$_\text{A}$ = 0.4\\
    (d) T and x$_\text{A}$ at 65 kPa and y$_\text{A}$ = 0.4\\
    (e) P and x$_\text{A}$ at 327 K and y$_\text{A}$ = 0.4\\
    (f) The fraction of the system that is liquid and the composition of the liquid and vapour in equilibrium at 327 K and 65 kPa when the overall composition of the system is 70 mole per cent acetone.
    
    \item Mixtures of n-Heptane (A) and n-Octane (B) are expected to behave ideally. The total pressure over the system is 101.3 kPa. Using the vapour pressure data given below,
    
    \begin{table}[H]
    \centering
    \begin{tabular}{|c|c|c|}
    \hline
    \multirow{2}{*}{\textbf{\begin{tabular}[c]{@{}c@{}}Temperature\\ T ($^{\circ}$C)\end{tabular}}} & \multicolumn{2}{c|}{\textbf{Vapour pressure (kPa)}} \\ \cline{2-3} 
      & \textbf{n-Hexane, \textit{P}$_\text{A}^\text{s}$} & \textbf{n-Octane, \textit{P}$_\text{B}^\text{s}$} \\ \hline
    371.4 & 101.3                 & 44.4                  \\ \hline
    378   & 125.3                 & 55.6                  \\ \hline
    383   & 140.0                 & 64.5                  \\ \hline
    388   & 160.0                 & 74.8                  \\ \hline
    393   & 179.9                 & 86.6                  \\ \hline
    398.6 & 205.3                 & 101.3                 \\ \hline
    \end{tabular}%
    \end{table}
    (a) Construct the boiling point diagram and\\
    (b) The equilibrium diagram and \\
    (c) Deduce an equation for the equilibrium diagram using an arithmetic average value of relative volatility.
    
    
    \item Binary system of acetonitrile (A) and nitromethane (B) conforms closely to Raoult's law. Vapour pressures for the pure species are given by the following Antoine equations.
    $$ln \  \text{\textit{P}}_\text{A}^S = 14.2724 - \frac{2945.47}{T-49.15}$$
    $$ ln \  \text{\textit{P}}_\text{B}^S = 14.2043 - \frac{2972.64}{T-64.15}$$
    \\where T is in K and \textit{P}$^\text{s}$ is in kPa. Prepare a graph showing T vs x$_\text{A}$ and T vs y$_\text{A}$ for a pressure of 70 kPa.
\end{enumerate}
\end{document}
